%% start of file `template.tex'.
%% Copyright 2006-2013 Xavier Danaux (xdanaux@gmail.com).
%
% This work may be distributed and/or modified under the
% conditions of the LaTeX Project Public License version 1.3c,
% available at http://www.latex-project.org/lppl/.

%\documentclass{report}
\documentclass[11pt,a4paper,sans]{moderncv}
\usepackage[utf8]{inputenc}
%\usepackage[frenchle]{babel}
\usepackage[francais]{babel}
\usepackage{geometry}
%\usepackage[frenchle]{babel}
%\usepackage[francais]{babel}        % possible options include font size ('10pt', '11pt' and '12pt'), paper size ('a4paper', 'letterpaper', 'a5paper', 'legalpaper', 'executivepaper' and 'landscape') and font family ('sans' and 'roman')

%\makeatletter
%\newcommand*{\extra}[2]{\addtofooter{\@extrainfo #1}\addtofooter{\@extrainfo #2}}
%\makeatother
\geometry{hmargin=2.5cm,top=0.5cm, bottom = 2.5cm, footnotesep = 1cm}

% moderncv themes
\moderncvstyle{casual}                             % style options are 'casual' (default), 'classic', 'oldstyle' and 'banking'
\moderncvcolor{blue} 
                              % color options 'blue' (default), 'orange', 'green', 'red', 'purple', 'grey' and 'black'
%\renewcommand{\familydefault}{\sfdefault}         % to set the default font; use '\sfdefault' for the default sans serif font, '\rmdefault' for the default roman one, or any tex font name
%\nopagenumbers{}                                  % uncomment to suppress automatic page numbering for CVs longer than one page
% character encoding
%\usepackage[utf8]{inputenc}                       % if you are not using xelatex ou lualatex, replace by the encoding you are using
%\usepackage{CJKutf8}                              % if you need to use CJK to typeset your resume in Chinese, Japanese or Korean

% adjust the page margins
%\usepackage{fullpage}
%\geometry{hmargin=2.5cm,topmargin=2.5cm}
%\addtolength{\voffset}{-2cm}
%\addtolength{\textheight}{12cm}
\geometry{scale=0.75}%\setlength{\hintscolumnwidth}{3cm}                % if you want to change the width of the column with the dates
%\setlength{\makecvtitlenamewidth}{3cm}           % for the 'classic' style, if you want to force the width allocated to your name and avoid line breaks. be careful though, the length is normally calculated to avoid any overlap with your personal info; use this at your own typographical risks...

% personal data
\name{ Norman \newline}{JAN - LE CLOIREC pouet}
\title{Recherche de stage de fin d'études \newline \hspace*{\fill} d'une durée de 6 mois \newline \hspace*{\fill} du 1 juillet à fin décembre}                               % optional, remove / comment the line if not wanted
\address{9 ``Le Cosquer''}{56 300 Malguénac}{France}% optional, remove / comment the line if not wanted; the "postcode city" and "country" arguments can be omitted or provided empty
\phone[mobile]{+336 (06) 42 99 83 62}                   % optional, remove / comment the line if not wanted; the optional "type" of the phone can be "mobile" (default), "fixed" or "fax"
%\phone[fixed]{+2~(345)~678~901}
%\phone[fax]{+3~(456)~789~012}
\email{norm.janlec@orange.fr}                               % optional, remove / comment the line if not wanted
%\homepage{www.johndoe.com}                         % optional, remove / comment the line if not wanted
%\social[linkedin]{john.doe}                        % optional, remove / comment the line if not wanted
%\social[twitter]{jdoe}                             % optional, remove / comment the line if not wanted
%\social[github]{jdoe}                              % optional, remove / comment the line if not wanted
\extrainfo{Permis B\footersymbol Véhiculé}%\\SST}                 % optional, remove / comment the line if not wanted
\photo[64pt][0.4pt]{moi_NB.png}                       % optional, remove / comment the line if not wanted; '64pt' is the height the picture must be resized to, 0.4pt is the thickness of the frame around it (put it to 0pt for no frame) and 'picture' is the name of the picture file
%\quote{Some quote}                                 % optional, remove / comment the line if not wanted

% to show numerical labels in the bibliography (default is to show no labels); only useful if you make citations in your resume
%\makeatletter
%\renewcommand*{\bibliographyitemlabel}{\@biblabel{\arabic{enumiv}}}
%\makeatother
%\renewcommand*{\bibliographyitemlabel}{[\arabic{enumiv}]}% CONSIDER REPLACING THE ABOVE BY THIS

% bibliography with mutiple entries
%\usepackage{multibib}
%\newcites{book,misc}{{Books},{Others}}
%----------------------------------------------------------------------------------
%            content
%----------------------------------------------------------------------------------
\begin{document}
%\begin{CJK*}{UTF8}{gbsn}                          % to typeset your resume in Chinese using CJK
%-----       resume       ---------------------------------------------------------
\makecvtitle
\vspace*{-1cm}
\section{\'Etudes - Formation}
\cventry{2014--2016}{\'Elève Ingénieur}{\'Ecole nationale d'Ingénieurs de Brest}{}{}{}%{City}{\textit{Grade}}{Description}
\cventry{2012--2014}{Cycle préparatoire}{\'Ecole nationale d'Ingénieurs de Brest}{}{}{}%{Description}  % arguments 3 to 6 can be left empty
\cventry{2012}{Baccalauréat}{Scientifique, option Sciences de l'Ingénieur}{Pontivy}{mention Bien}{}{}

\section{Expériences professionnelles}
\subsection{Liées à la formation}



\cventry{Janvier 2014}{Stage de découverte de l'entreprise}{LINPAC}{Noyal-Pontivy-56920}{}{Service Coextrusion}
\subsection{Autres}

\cventry{étés 2013--2014}{Opérateur}{LINPAC}{Noyal-Pontivy-56920}{}{Service Automatic Side Cutting}



\section{Compétences}
\setlength{\hintscolumnwidth}{1cm} 
\begin{cvcolumns}

  
	
	\cvcolumn{Mécanique}{\begin{itemize}\item Dessin industriel\item Analyse critique d'un système \item
		Calculs d'efforts sur un système \item Résistance des matériaux \item Thermique\end{itemize} \textit{CATIA, RDM le Mans}}
	
	\cvcolumn{Automatisme}{\begin{itemize}\item Connaissance des composants usuels de puissance\item GRAFCET \item Sécurisation des systèmes automatisés \item Supervision\end{itemize} \textit{PC-vue, PL7}}

  \cvcolumn{Électronique}{Analogique\begin{itemize}\item Dimensionnement de composants\item Filtrage séquentiel\item Analyse et traitement de signal\end{itemize}Numérique\begin{itemize}\item Circuits logiques (combinatoires et séquentiels)\item Microprocesseur\end{itemize} \textit{LTSpice, Proteus}}
	
	\end{cvcolumns}


\setlength{\hintscolumnwidth}{3cm} 
\section{Informations complémentaires}
\setlength{\hintscolumnwidth}{1cm} 
%\hspace*{-3cm}
%\cvitemwithcomment{Anglais}{TOEIC}{B2 en cours d'acquisition}
%\cvitemwithcomment{Allemand}{}{Lu, parlé, écrit}
\hspace*{-3cm}\begin{cvcolumns}

	
	\cvcolumn{Centres d'intérêts}
{\begin {description} 
\item{}{ Cinéma}
\item{}{ Musique} 
\item{}{ VTT} 
\item{}{ Bricolage}\end{description}}
	%\newline \hspace{2mm}B2 en cours d'acquisition
	\cvcolumn{Langues} {\cvitemwithcomment{Anglais}{\hspace{0.5mm} \textmd{TOEIC en cours d'acquisition}}\noindent{\hspace{1.6cm}B2}
\cvitemwithcomment{Allemand}{\hspace{2mm}\textmd{Lu, parlé, écrit}}{}}

  
	\end{cvcolumns}



%\section{Centres d'intérêts}


%\cvitem{}{ Cinéma}
%\cvitem{}{ Musique} 
%\cvitem{}{ VTT} 
%\cvitem{}{ Bricolage}





% Publications from a BibTeX file without multibib
%  for numerical labels: \renewcommand{\bibliographyitemlabel}{\@biblabel{\arabic{enumiv}}}% CONSIDER MERGING WITH PREAMBLE PART
%  to redefine the heading string ("Publications"): \renewcommand{\refname}{Articles}
%\nocite{*}
%\bibliographystyle{plain}
%\bibliography{publications}                        % 'publications' is the name of a BibTeX file

% Publications from a BibTeX file using the multibib package
%\section{Publications}
%\nocitebook{book1,book2}
%\bibliographystylebook{plain}
%\bibliographybook{publications}                   % 'publications' is the name of a BibTeX file
%\nocitemisc{misc1,misc2,misc3}
%\bibliographystylemisc{plain}
%\bibliographymisc{publications}                   % 'publications' is the name of a BibTeX file


%-----       letter       ---------------------------------------------------------
% recipient data
         % use an optional argument to use a string other than "Enclosure", or redefine \enclname

%\clearpage\end{CJK*}                              % if you are typesetting your resume in Chinese using CJK; the \clearpage is required for fancyhdr to work correctly with CJK, though it kills the page numbering by making \lastpage undefined
\end{document}


%% end of file `template.tex'.
